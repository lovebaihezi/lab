\documentclass[UTF8]{ctexart}

\usepackage[british,UKenglish]{babel}
\usepackage{amsmath}
\usepackage{titlesec}
\usepackage{color}
\usepackage{graphicx}
\usepackage{fancyref}
\usepackage{hyperref}
\usepackage{float}
\usepackage{scrextend}
\usepackage{setspace}
\usepackage{xargs}
\usepackage{multicol}
\usepackage{nameref}
\usepackage{xeCJK}
\usepackage{sectsty}
\usepackage{multicol}
\usepackage{multirow}
\usepackage[procnames]{listings}
\usepackage{appendix}
\usepackage{listings}

\newcommand\tab[1][1cm]{\hspace*{#1}}
\hypersetup{colorlinks=true, linkcolor=black}
\interfootnotelinepenalty=10000

\newcommand{\cleancode}[1]{\begin{addmargin}[2em]{2em}\texttt{\textcolor{cleanOrange}{#1}}\end{addmargin}}
\newcommand{\cleanstyle}[1]{\text{\textcolor{cleanOrange}{\texttt{#1}}}}

\setCJKmainfont{Source Han Sans CN Medium}

\usepackage[colorinlistoftodos,prependcaption,textsize=footnotesize]{todonotes}
\newcommandx{\commred}[2][1=]{\textcolor{Red}
{\todo[linecolor=red,backgroundcolor=red!25,bordercolor=red,#1]{#2}}}
\newcommandx{\commblue}[2][1=]{\textcolor{Blue}
{\todo[linecolor=blue,backgroundcolor=blue!25,bordercolor=blue,#1]{#2}}}
\newcommandx{\commgreen}[2][1=]{\textcolor{OliveGreen}{\todo[linecolor=OliveGreen,backgroundcolor=OliveGreen!25,bordercolor=OliveGreen,#1]{#2}}}
\newcommandx{\commpurp}[2][1=]{\textcolor{Plum}{\todo[linecolor=Plum,backgroundcolor=Plum!25,bordercolor=Plum,#1]{#2}}}

\def\code#1{{\tt #1}}

\def\note#1{\noindent{\bf [Note: #1]}}

\makeatletter
%% The "\@seccntformat" command is an auxiliary command
%% (see pp. 26f. of 'The LaTeX Companion,' 2nd. ed.)
\def\@seccntformat#1{\@ifundefined{#1@cntformat}%
   {\csname the#1\endcsname\quad}  % default
   {\csname #1@cntformat\endcsname}% enable individual control
}
\let\oldappendix\appendix %% save current definition of \appendix
\renewcommand\appendix{%
    \oldappendix\newcommand{\section@cntformat}{\appendixname~\thesection\quad}
}
\makeatother

\lstdefinelanguage{Julia}%
  {morekeywords={abstract,break,case,catch,const,continue,do,else,elseif,%
      end,export,false,for,function,immutable,import,importall,if,in,%
      macro,module,otherwise,quote,return,switch,true,try,type,typealias,%
      using,while,|>, |>, =>, ->},%
   sensitive=true,%
   morecomment=[l]\#,%
   morecomment=[n]{\#=}{=\#},%
   morestring=[s]{"}{"},%
   morestring=[m]{'}{'},%
}[keywords,comments,strings]%

\lstset{%
    language         = Julia,
    basicstyle       = \fontfamily{Fira Code},
    keywordstyle     = \bfseries\color{blue},
    stringstyle      = \color{magenta},
    commentstyle     = \color{ForestGreen},
    showstringspaces = false,
}
\lstdefinelanguage{Nasm}%
  {morekeywords={mov,eax,ebx,cmp,je,jmp,ret,add,sub,global},%
   sensitive=true,%
   morecomment=[l]\;\;,%
   morestring=[s]{"}{"},%
   morestring=[m]{'}{'},%
}[keywords,comments,strings]%

\lstset{%
    language         = Nasm,
    basicstyle       = \fontfamily{Fira Code},
    keywordstyle     = \bfseries\color{blue},
    stringstyle      = \color{magenta},
    commentstyle     = \color{green},
    showstringspaces = false,
}
\lstset{frame=, basicstyle={\footnotesize\ttfamily}}



\graphicspath{ {images/} }
\usepackage{ctex}
\usepackage{fontspec}
\setmonofont[
  Contextuals={Alternate}
]{Fira Code}

\begin{document}
\renewcommand{\contentsname}{目\ 录}
\renewcommand{\appendixname}{附录}
\renewcommand{\appendixpagename}{附录}
\renewcommand{\refname}{参考文献} 
\renewcommand{\figurename}{图}
\renewcommand{\tablename}{表}
\renewcommand{\today}{\year\month\day}

\title{{\Huge 计算机组成原理实验报告{\large\linebreak\\}}{\Large 实验一: 算数逻辑运算实验\linebreak\linebreak}}
\author{\\姓\ 名:柴\ 博\ 文\\
学\ 号: 04194012\\
班\ 号: 大数据1901\\\\
计算机组成原理\\
(秋季, 2021)\\\\
西安邮电大学\\
计算机学院\\
数据科学与大数据专业}
\date{\today}
\newpage

\begin{center}
{\Large\bf{摘\ 要\\}}
\end{center}

本次实验代码均可以在\href{https://github.com/lovebaihezi/lab/tree/main/Asm/riscv}{github仓库}下找到.

本次实验使用Nasm 语言,通过Nasm进行编译,通过C编译器进行链接编译至可执行文件

\newpage
\begin{center}
\tableofcontents\label{c}
\end{center}

\newpage

\section{实验过程}\label{overview}
\subsection{实验所用指令}
首先是本次实验使用到的指令:
\begin{table}[h]
  \centering
  \begin{tabular}{c|c}
    指令 & 描述 \\hline
    mov  & 将,后的值移入前面的存储区当中\\
    add  & 将前后两个的值之和移入,前面的存储区当中\\
    cmp  & 比较,两侧的数字,将结果存储在比较寄存器当中\\
    jle  & 如果比较寄存器当中的结果为小于等于,则跳转至标签处\\
    main & 标准C程序入口\\
  \end{tabular}
\end{table}
\begin{lstlisting}{lang=Nasm}
global main

;; 1 2 3 4 5 6 7  8  9  10
;; 1 1 2 3 5 8 13 21 34 55

main:
    mov eax, 1;; 将EAX寄存器的值设为1
    mov ebx, 1
    mov ecx, 9
fib:
    cmp ecx, 1;;比较ecx中的值和1比谁大
    je end;;相等就推出
    mov edx, ebx;;edxu用来交换eax和ebx的值
    add ebx, eax
    mov eax, edx
    sub ecx, 1
    jmp fib;;形成循环
end:
    mov eax, ebx;;循环终止 将得到的结果移给eax然后程序退出
    ret

\end{lstlisting}
\end{document}
