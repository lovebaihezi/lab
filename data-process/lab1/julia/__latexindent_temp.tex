\documentclass[a4paper]{article}

% \usepackage{inputenc}
\usepackage[british,UKenglish]{babel}
\usepackage{amsmath}
\usepackage{titlesec}
\usepackage{color}
\usepackage{graphicx}
\usepackage{fancyref}
\usepackage{hyperref}
\usepackage{float}
\usepackage{scrextend}
\usepackage{setspace}
\usepackage{xargs}
\usepackage{multicol}
\usepackage{nameref}

\usepackage{sectsty}
\usepackage{multicol}
\usepackage{multirow}
\usepackage[procnames]{listings}
\usepackage{appendix}
\usepackage{listings}

\newcommand\tab[1][1cm]{\hspace*{#1}}
\hypersetup{colorlinks=true, linkcolor=black}
\interfootnotelinepenalty=10000

\newcommand{\cleancode}[1]{\begin{addmargin}[3em]{3em}\texttt{\textcolor{cleanOrange}{#1}}\end{addmargin}}
\newcommand{\cleanstyle}[1]{\text{\textcolor{cleanOrange}{\texttt{#1}}}}


\usepackage[colorinlistoftodos,prependcaption,textsize=footnotesize]{todonotes}
\newcommandx{\commred}[2][1=]{\textcolor{Red}
{\todo[linecolor=red,backgroundcolor=red!25,bordercolor=red,#1]{#2}}}
\newcommandx{\commblue}[2][1=]{\textcolor{Blue}
{\todo[linecolor=blue,backgroundcolor=blue!25,bordercolor=blue,#1]{#2}}}
\newcommandx{\commgreen}[2][1=]{\textcolor{OliveGreen}{\todo[linecolor=OliveGreen,backgroundcolor=OliveGreen!25,bordercolor=OliveGreen,#1]{#2}}}
\newcommandx{\commpurp}[2][1=]{\textcolor{Plum}{\todo[linecolor=Plum,backgroundcolor=Plum!25,bordercolor=Plum,#1]{#2}}}

\def\code#1{{\tt #1}}

\def\note#1{\noindent{\bf [Note: #1]}}

\makeatletter
%% The "\@seccntformat" command is an auxiliary command
%% (see pp. 26f. of 'The LaTeX Companion,' 2nd. ed.)
\def\@seccntformat#1{\@ifundefined{#1@cntformat}%
   {\csname the#1\endcsname\quad}  % default
   {\csname #1@cntformat\endcsname}% enable individual control
}
\let\oldappendix\appendix %% save current definition of \appendix
\renewcommand\appendix{%
    \oldappendix
    \newcommand{\section@cntformat}{\appendixname~\thesection\quad}
}
\makeatother

\lstdefinelanguage{Julia}%
  {morekeywords={abstract,break,case,catch,const,continue,do,else,elseif,%
      end,export,false,for,function,immutable,import,importall,if,in,%
      macro,module,otherwise,quote,return,switch,true,try,type,typealias,%
      using,while,|>, .|>, =>, ->},%
   sensitive=true,%
   % alsoother={$},%
   morecomment=[l]\#,%
   morecomment=[n]{\#=}{=\#},%
   morestring=[s]{"}{"},%
   morestring=[m]{'}{'},%
}[keywords,comments,strings]%

\lstset{%
    language         = Julia,
    basicstyle       = \fontfamily{Fira Code},
    keywordstyle     = \bfseries\color{blue},
    stringstyle      = \color{magenta},
    commentstyle     = \color{ForestGreen},
    showstringspaces = false,
}




\lstset{frame=, basicstyle={\footnotesize\ttfamily}}



\graphicspath{ {images/} }
\usepackage{ctex}
%-----------------------------------------BEGIN DOC----------------------------------------

\begin{document}
\renewcommand{\contentsname}{目\ 录}
\renewcommand{\appendixname}{附录}
\renewcommand{\appendixpagename}{附录}
\renewcommand{\refname}{参考文献} 
\renewcommand{\figurename}{图}
\renewcommand{\tablename}{表}
\renewcommand{\today}{\number\year 年 \number\month 月 \number\day 日}

\title{{\Huge 数据挖掘实验实验报告{\large\linebreak\\}}{\Large 实验一: 数据预处理\linebreak\linebreak}}
%please write your name, Student #, and Class # in Authors, student ID, and class # respectively
\author{\\姓\ 名:柴\ 博\ 文\\ 
学\ 号: 04194012\\
班\ 号: 大数据1901\\\\
数据挖掘与机器学习\\
(秋季, 2021)\\\\
西安邮电大学\\
计算机学院\\
数据科学与大数据专业}
\date{\today}
\maketitle
\newpage

%-----------------------------------------ABSTRACT-------------------------------------
\begin{center}
{\Large\bf{摘\ 要\\}}
\end{center}
本次实验使用Julia语言进行实现.

实验报告采用LaTeX, 在overleaf上进行编写.

通过DataFrames, CSV, XLSX读取数据, PyPlots, Plots, StatsPlot绘制图案.

本次实验代码均可以在\href{https://github.com/lovebaihezi/lab/tree/main/data-process/lab1/julia}{github仓库}下找到.

\newpage
%-----------------------------------------CONTENT-------------------------------------
\begin{center}
\tableofcontents\label{c}
\end{center}
\newpage

%------------------------------------------TEXT--------------------------------------------

%----------------------------------------OVERVIEW-----------------------------------------

\section{概述} \label{overview}%------------------------------

1、掌握数据探索统计特征计算、数据可视化等基本方法

2、掌握数据集缺失值、含噪数据的平滑处理、数据变换、数据集成等预处理方法。

3、掌握PCA主成分分析等降维方法

\begin{itemize}
	\item{\textbf{数据可视化}对某县广电宽带用户的5000条数据(或者自己感兴趣的其他领域的数据)进行探索,通过统计特征可视化进行数据分析,探索发现你感兴趣的知识。}
    \item{\textbf{数据处理}对北京西安的年薪数据(或者自己感兴趣的其他领域的数据)计算均值,方差等统计特征,绘制据箱体图和小提琴图等图,分析北京西安年薪的差异。}
    \item{\textbf{数据清洗}用'movie\_metadata.csv'数据集(或者自己感兴趣的其他领域的数据)进行案例分析,这个数据集包含了包括演员、导演、预算、总输入,以及IMDB评分和上映时间等信息,进行处理缺失数据,可以是添加默认值,删除不完整的行,异常值处理,重复数据处理,规范化数据类型等等。
    }
    \item{\textbf{数据集成}合并两个给定数据集:ReaderRentRecode.csv和ReaderInformation.csv(或者自己感兴趣的其他领域的数据),其中两个数据集的共同点是具有相同的num属性,最终生成一个综合的数据集。
    }
    \item{\textbf{PCA}使用鸢尾花数据集(或者自己感兴趣的其他领域的数据),这个数据集有150个样本,其中每个样本有五个变量,其中四个为特征变量,分别为萼片长度(Sepal length), 萼片宽度(Sepalwidth),花瓣长度(Petallength),花瓣宽度(Petalwidth),还有一个变量是其所属的品种的类别变量(Species),这个鸢尾花内别共有3种类别分别是山鸢尾(Iris-setosa)、变色鸢尾(Iris-versicolor)和维吉尼亚鸢尾(Iris-virginica),首先对4维的原始数据集实现可视化,可视化一组数据来观察数据分布,然后对数据集进行标准化(归一化),接着利用PCA主成分分析将数据降到二维。}
\end{itemize}

%------------------------------------Lab Process--------------------------------------

\newpage
%------------------------------
\section{数据可视化}\label{sub:ptx}
\subsection{实验过程} \label{sub:ptxproc}

首先讲旧版Excel格式的xls文件转换为CSV文件\href{"https://github.com/lovebaihezi/lab/blob/main/data-process/lab1/julia/file/xian_guangdian.csv"}{github}

随后使用CSV读取文件内容, 并通过DataFrame解析格式以及类型

随后将数据根据客户等级进行分组,总共有

\begin{lstlisting}[language=julia]
    function free()
        file_path = "lab1/julia/file/xian_guangdian.csv"
        data = file_path |> CSV.File |> DataFrame
        data = groupby(select(data, "客户等级")
        quality = combine(nrow, , "客户等级"))
        StatsPlots.bar(quality[!, 1], quality[!, 2], label = "count")
    end
\end{lstlisting}

图\ref{fig:singleblock}. 

\begin{figure}[ht]
 \centering
 \includegraphics[height=12cm]{images/广电信息CSV展示.png}
 \caption{广电信息图}
 \label{fig:singleblock}
\end{figure}

\subsection{实验结果和分析}\label{sub:ptxeva}
我们可以从终端中看到结果:
城市用户:2968
农村用户:2026
同时可以从图像上看出办理宽带的城市用户多余农村用户
% --------------------------------Evaluation----------------------------------------
\section{PTX与x86指令的比较}\label{ptxvsx86}
以下请按照上面的说明示例,自行安排章节内容。

% -----------------------------------Appendix----------------------------------------
\appendix
\section{代码}\label{sub:app.code}
请在附录\ref{sub:app.code}中添加代码。请使用如下C或者C++的语法高亮描述方法。
\begin{lstlisting}[language=julia]

using XLSX;
using CSV;
using DataFrames;
using Plots;
using Dates;
using StatsPlots;
using PyPlot;

file_path = "file/xian_guangdian.csv";
data = CSV.File(file_path) |> DataFrame


\end{lstlisting}
\newpage
% -----------------------------------REFERENCE----------------------------------------
\begin{thebibliography}{9}
\bibitem{Erdos01} P. Erd\H os, \emph{A selection of problems and
results in combinatorics}, Recent trends in combinatorics (Matrahaza,
1995), Cambridge Univ. Press, Cambridge, 2001, pp. 1--6.
\end{thebibliography}
\end{document}

